\documentclass[a4paper,11pt]{article}
%\documentclass[a4paper,draft]{jpconf}

\input csz

\usepackage{fullpage}
\usepackage{multicol}
\usepackage{graphicx}
\usepackage{url}
\usepackage{xspace}

\def\smalltt#1{{\small\texttt{#1}}}
\def\foottt#1{{\footnotesize\texttt{#1}}}

\def\q{\quad}

\def\tightlist{\setlength{\topsep}{0pt}\setlength{\itemsep}{0pt}}

\def\greed{\textsc{Greed}\xspace}
\def\greedhome{\textsc{Greed}@\texttt{home}\xspace}
\def\root{{ROOT}\xspace}
\def\xrootd{\texttt{xrootd}\xspace}
\def\aliroot{{AliROOT}\xspace}
\def\alien{{AliEn}\xspace}
\def\gled{\textsc{Gled}\xspace}
\def\web{\texttt{web}\xspace}
\def\grid{\texttt{grid}\xspace}
\def\bittorrent{\texttt{BitTorrent}\xspace}


\title{Greed Manifesto}

\author{Matev\v{z} Tadel}

\date{\today}


\begin{document}

\maketitle

\setcounter{section}{-1}


\section{Document status}

\paragraph{v1.0, Oct--Dec 2007.}
Initial version intended for friends. Presents a dream of the \greed
project and wanders off into the realms of eventual possibilities.

To be sent to about 20 people.

\paragraph{v1.0.1, 20 Jan 2008.}
Incorporated comments from Jona.


% Family: Alja, Vid
%
% Friends: Jona, Darko, Klemen, Ziga, Iggy, Siso, Joh
%
% IJS: Mare, Kerso, Filipko, Miso, Cric, Johhny
%
% ALICE / ROOT: Federico, Bertrand, Axel, Fons, Rene, Predrag,
% (? Andrei, Cvetan, Peter, Latchezar, Jan-Fiete, Kostin)
% Derek, Andreas,
% Timur, Artem,
% Jochen, Volker,
%
% CMS: Chris, Andrea, Avi


% ========================================================================

\section{Introduction}

\greed is a vehicle for dissemination and universal exploitation of
scientific and educational programs. It combines the concepts of
\grid, \texttt{computing@home} and massive multi-player online games
with the \emph{free and open-source software} ideology and development
practices. The main aim of \greed is to become a link between
institutions for research \& education and Internet users by providing
a global and open computing environment that allows all parties to
pursue their own goals and interests. Institutions utilize the
computing resources of Internet users and in return provide a
high-reliability server infrastructure needed for operation of
persistent virtual worlds created and maintained by communities of
\greed members.

In its essence, \greed is a hack to make this world a better place.
It attempts to expose greed both as the ultimate source of motivation
as well as an unwholesome tendency that limits the human cooperation
and hinders global development. But greed is not a very good name for
a project, unless it is an acronym: \greed is a \emph{Global Research
  Environment for Equitable Development}.\footnote{\emph{Global
    Research Environment for Eidetic Design} is also an option,
  suggestions are welcome.}


% ========================================================================

\section{Roots of the \greed initiative}

Every global initiative starts with a concrete problem and with a
vision of how the present elements can be combined for the common
good.  \greed is no exception to this rule: it is motivated by the
computing needs of the \emph{Large Hadron Collider} (LHC) experiments
and by the desire of its individual members to expose the computing
technologies developed for operation of LHC and use them as an
outreach stunt for popularization of scientific theories and
contemporary technological endeavors undertaken for their exploration.

While the basic computing requirements of the LHC experiments are
covered by the dedicated infrastructure of the LHC computing \grid,
there is little spare capacity for advanced searches for exotic
particles and new physics which require significant computing
resources, mainly in terms of event simulation and analysis of both
simulated and real experiment data.\footnote{The resources provided to
  the ALICE experiment during the first four years of the detector
  operation are actually significantly below the pledged level.}
The additional computing resources provided by the public at large
could thus contribute the final stone to the full exploitation of the
physics potential of the LHC.

As the full experiment software and accompanying data would have to be
distributed for the operation of such infrastructure, it would also
become feasible for anybody to run simple data visualization and
analysis applications allowing better understanding of the LHC
project, of the LHC experiments and of high-energy physics in general.
This is of particular interest for universities and high-schools as
the programs could be used for student projects or for exercises in
physics, computing and statistics. Additionally, educational
institutions usually own a fair number of computers (desktops in
offices and computing rooms) that are only utilized half of the time
as well as have a reasonable WAN up-link and can thus also offer CPU
power for utilization by the LHC experiments.

On the other side of the equation we have 1.2 billion Internet
users that we want to inform about and engage in the latest scientific
and technological developments in an attractive and accessible manner.
This calls for an analysis of the possible user base, keeping in mind
that another purpose of the \greed is to harvest the computing
resources of its target audience for scientific computation.


\subsection{Possible audiences of the \greed}

The success of several \smalltt{computing@home} projects during the
last five years\footnote{Currently folding@home is the most active one,
  just passing the tera-flop mark with 500k registered users. During
  the last year PS3 clients accounted for a better part of the
  project's growth.} points to a surprisingly high level of interest
for scientific computing present among the general public. The basic
terms of contract between the users and the project seem, in most
cases, deceptively simple. The project provides a computing platform
and a basic community portal that often consists mostly of
expert-level information (references to scientific papers or research
laboratory web-pages). Based on this rudimentary infrastructure, the
users connect into the distributed computing environment, contributing
their CPU, disk-space and bandwidth for no apparent benefit. But
people who are willing to go through the trouble of installing the
computing client and paying for the electricity\footnote{Estimating
  power consuption to 100W per core, the monthly consumption is 48kWh.
  For France this makes about 6.5 EUR per month per core.} more than
likely care about what they are contributing to, even though this can
be any of the following things:
\begin{itemize}\tightlist
\item rational exploitation of existing computing resources;
\item development and utilization of cool new software and ideas
  pertaining to networks, computing and information technologies in
  general (e.g. FOSS movement, \grid computing, peer-to-peer
  computing);
\item supporting a specific project due to its perceived
  scientific or humanitarian merit.
\end{itemize}
Whatever are the actual motives, the contributors are effectively
forming a community and many of them would feel inclined towards
investing some of their time into further exploration of the project
they are contributing to. The idea of \greed encompasses all of the
above motivations and it would almost certainly appeal to the majority
of existing contributors.

At this point let us also consider the popularity of community
knowledge sites (Wikipedia), of alternate reality worlds (SecondLife)
and of massive multi-player online games (World of Warcraft, EVE
Online). Most of the services in the last two categories also require
a subscription with monthly charges up to 25\$. There are two
common attributes of users who participate in these activities that
also make them possible \greed contributors:
\begin{itemize}\tightlist
\item they have a PC with Internet access and use it as a communication
  engine, as a hobby, or simply for fun;
\item they spend significant amounts of their time online and are
  willing to contribute their work or effort to the community.
\end{itemize}
The combined user bases of these projects well exceed 20 million
people. But unlike the contributors to \smalltt{computing@home}
projects, the users of these community-engines do not have a common
set of motives. Thus either a sub-set of of users needs to be selected
or, alternatively, a variety of strategies need to be employed to
arouse the interest of as many users as possible.

To summarize, there are many existing activities on the Internet that
provided aspiration for \greed. One could even say that \greed is a
peculiar combination of elements from the projects mentioned in this
section that have not been placed into a common context yet.


\subsection{High-level description of \greed worlds}

Persistent multi-user virtual worlds have been chosen as the primary
means of dissemination because they appeal to a variety of audiences,
from primary school level onwards, determined by the complexity and
scope of the virtual world elements. To maintain coherency and allow
information exchange between different worlds, a common basic world
structure must be employed. It should be general enough to allow a
multitude of scientific and technological elements to be combined into
coherent worlds of different characters. At the same time, within
the mechanics of each such world, it must be possible for users to
participate in the world development on completely different levels of
engagement and personal time investment.

To achieve that, the worlds are presentable on different levels. Each
presentation offers a specific view of the world and offers a
specialized user interface for interaction with the presented world
elements.
\begin{itemize}\tightlist

\item \emph{Strategic} view provides insight into relations among
  entities on a large-scale by presenting status and activity of
  large, compound world entities and their communications. This
  information can be combined with maps or diagrams of extensive
  sections of space, like galactic regions, solar or planetary
  systems, or individual planets.  When presenting a specific element,
  e.g., production, distribution and consumption of a valuable
  resource, the relation among contributing entities can be visualized
  as a 3-dimensional graph, supplemented by histograms, plots and
  tables. High-level settings can be changed here, affecting a
  multitude of entities by effectively altering the parameters of
  world development algorithms.

\item \emph{Operational} view focuses on specific tasks or on
  relations among entities in a more local context. For example, this
  view would be used for planning and execution of a
  resource-discovery mission followed by construction of
  resource-extraction and transport facilities. Settings from the
  operational level influence a specific group of entities via
  interaction with the control agents, not with the individual
  entities. As above, map and 3D-graph views can be employed. In this
  mode it is also possible to use an immersive 3D-view, augmented with
  overlays of schematic data.

\item \emph{Tactical} view presents environment and world entities in
  detail and allows management of individual entities by assigning them
  specific tasks or control agents. Map and augmented 3D views are
  used here, possibly accompanied by a radar-like view of the
  operational area.

\item \emph{Entity} view focuses on different aspects of control over
  an individual entity. This ranges from first-person view and
  operation of a vehicle or a space-ship to control over production
  parameters of a factory or laboratory.

\end{itemize}
Strategic and operational views can focus on different world aspects,
e.g., on transport routes, on energy acquisition and distribution
networks, on world exploration and surveillance, or on progress in
science and engineering.

The initial layout of the virtual worlds follows the structure of a
solar system\footnote{Different topologies with similar consequences
  for the world dynamics are also possible, e.g. \emph{Emental} world
  consisting of a number of Dyson's spheres connected by a set of
  stable tubular wormholes.} and prompts users to engage in
exploration, resource acquisition, infrastructure construction and
research activities by controlling the in-world entities. The initial
time is set about 20 years into the future, with reasonably
extrapolated technologies and with the addition of two new fictitious
technologies: the quark-gluon plasma reactor and the Higgs-field
driven vacuum polarization chamber, both spin-offs of the LHC project,
allowing construction of a new generation of extreme energy sources,
control over large-scale quantum coherence and rudimentary
manipulation of the space-time continuum.

To underline the fragility of our civilization, the story could begin
with a cataclysm: human population of Earth is decimated by a global
disaster (caused by biological agents, comet impact, or global
redefinition of weather patterns) and the survivors are in position to
operate robotic factories and agents. The resources from Earth and
Moon are used to build Foundation vessels that are sent through poorly
controlled wormholes into random solar systems of our universe where
they have to bootstrap new civilizations. Of course, the users
themselves will determine for each individual world if its further
development will be carried out in a cooperative or in an opposing
manner.

Humans are a scarce and vulnerable resource in these conditions,
vitally needed for scientific and technological development as well as
for operation, maintenance and upgrades of existing technological
units. The human population dynamics and living conditions of the
humans determine the number of skilled scientists and engineers that
are produced within each simulated community and become available as
controllable world entities.

To actively employ such a specialist, the user must also spend
\emph{research credits} that can only be gained by providing computing
resources to \greed institutions. Alternatively, some research credit
can also be gained via online quizzes where a user is given a specific
data-set on which he must perform data-analysis and correctly reply to
a set of questions. The virtual-world research cost needed to achieve
major scientific and technological breakthroughs far exceeds the
computing resources of a single user, thus forcing the users to form
cooperative communities.

Introduction of new technologies and research paths is determined both
by the world maintainers and by the users' communities. Users
implicitly provide the development preferences by concentrating their
effort on specific areas while the world maintainers need to provide
adequate scientific and technological improvements. New technologies
will inevitably also introduce new world entities and provide new
requirements for extensions of the world mechanics. At this point the
developers of the core framework and world steering algorithms will
have to incorporate the new technology into the system and make sure
that it functions consistently with the existing world elements.

The in-world technologies introduced in the future will eventually
allow controlled long-range space travel, leading to possibilities for
further exploration and colonization as well as for establishment of
connections between different worlds, to increased cooperation and
possibly to conflict.


\subsection{Possible extensions of \greed world mechanics}

This section presents a review of different research fields that could
be introduced into \greed worlds either to increase their level of
detail or to imbue them with realistic scientific models and knowledge
databases. In this way \greed can provide a virtual environment that
is based on the latest research results and can be directly used not
only as a tool for teaching and for individual exploration of recorded
knowledge but also as an engine for simulation of complex systems. In
this spirit it is hoped that other research institutions and projects
will join the \greed initiative by providing their knowledge, manpower
and server infrastructure to receive the benefits of user-provided CPU
power and popularize their field of expertise among the general
public.

Astronomy and cosmology strive for the understanding of the universe
and of all the phenomena that proceed within it on a wide range of
spacetime-scales.  It is therefore necessary to include the
astronomical facts, at least at a basic level, into the world
mechanics from the very beginning.  With the interest from the
astro-physics community, detailed knowledge about universe could be
incorporated into the world mechanics, from realistic taxonomy and
composition of solar and planetary systems to large-scale structure of
galaxies, galactic clusters and beyond.

General theory of relativity plays an important role in the simulation
of long-range space travel and in presentation of exotic astronomic
phenomena while quantum mechanics and nano-technology present the
basic sciences for future technological development. Additionally,
hypothetical couplings between general relativity and quantum
mechanics could be stipulated and explored in semi-realistic
technological contexts leading to, for example, faster then light
travel and space-time manipulation devices.

Many disciplines of computer science naturally permeate all elements of
the world simulation software and of distributed information services
required for world operation and maintenance. Staying within the
aspects of the world mechanics and presentation, the most important
fields are computer graphics, computer vision, robotics, artificial
intelligence and control of autonomous agents. All of them can be
employed on both sides of the world, by the users to steer their
development and by the world maintainers to control non-interactive
world entities.

Resource acquisition, resource processing and industrial production
can be simulated by various aspects of geology, metallurgy, chemistry,
material science, engineering and process control. Futuro-realistic
simulation of energy generation and consumption provides an
interesting platform for exploration of alternative energy sources,
energy storage technologies and conversion devices. Construction and
operation of infrastructure, habitats and transport vehicles depends
on all of these disciplines, thus allowing the introduction of
complex world constructs, mechanisms and constraints.

Distribution of resources on the level of planetary and solar systems
determines the location of initial settlements and leads, as
technology progresses, to the need for safe and energy-efficient
transportation of raw-materials, products and world entities among
different communities. Trade and other forms of economic cooperation
naturally arise in such circumstances, adding a new level of activity
into the world mechanics. Of course, technology, knowledge and
expertise can likewise become the objects of trade.

Management of human population initially starts with a basic
population dynamics. As human specialists play an important role in
research, engineering and process control, the main emphasis is put
on education and general well being of human communities. Cultural
aspects can be introduced from two sides, from the direction of the
individual and from the direction of the society. In the first case
art and philosophy influence the possibilities of individuals for
attainment of knowledge, expertise and wisdom. In the second case,
human liberties, moral values, ideologies and religions determine the
level of social coercion and define the possible scope of social
manifestations as well as of reactions of an individual to the imposed
conditions. In an extreme case, these models could be used to explore
alternative forms of government, social solidarity and economic
relations with the goal of increasing human prosperity, stimulating
the growth of knowledge and maximizing the rate of technological
development.


\subsection{Levels of realism in \greed worlds}

Combination of all the above elements obviously leads to a world that
is overly complex and, as it seems at the moment, quite impossible to
construct and to maintain. While this remains the final goal, it seems
a sound tactics to start with several disparate worlds that are
constructed from a subset of carefully selected elements, chosen by
considering several aspects of the world simulation.
\begin{itemize}\tightlist
\item \emph{Scope and detail of scientific, technological and
    sociological elements} determines the active world mechanisms and
  sets the main roles available to users.
\item \emph{Intended public} directly influences the
  complexity of both the simulated world and world mechanisms.
\item \emph{Spatial extent, duration and tempo} pose the requirements
  for the server infrastructure. They also determine the amount of
  time users must invest into the interaction with the world to
  accomplish sizable advancement.
\end{itemize}
Users' feedback and interest in actual running worlds will help
to determine what types of worlds are of interest for different
audiences and will further influence the development of core world
simulation framework.

\paragraph{Basic worlds} are built around a small number of basic
elements. The world duration ranges from hours to weeks, the
spatial extent is limited accordingly. They have three main uses:
\begin{enumerate}\tightlist
\item worlds for new users that need to gain initial understanding of
  the world mechanisms and concepts or want to explore a new type of
  world;
\item worlds with content and complexity suitable for children (user
  interface can be simplified to the level of simple computer games);
  and
\item testing of new world mechanisms and their balancing against
  existing elements.
\end{enumerate}
As these worlds are relatively short-lived, they can be regularly
tailored to serve the changing needs of the community.

\paragraph{Intermediate worlds} combine more elements on a higher
level of detail and their duration ranges from months to years. These
worlds can focus on a variety of topics from the repertoire described
in previous sections and only experimentation can show which areas
will attract most users. As the chosen emphasis also determines the
level of users' engagement, several different worlds will be needed to
cover the expectations of individual users.

\paragraph{Realistic worlds} with all available complexity in certain
field can be used by specialists for testing of real-world algorithms,
for development of AI agents and for simulations of space exploration
missions. By keeping the user interface consistent with the one of
intermediate worlds, the users can intuitively enter them to gain
insight into current research or use them as a knowledge data-base
system.


\subsection{Conclusion}

Due to the multitude of directions the development might take, it
seems pointless to consider the final structure and content of the
worlds in a more detailed manner. However, it is important that the
actual flow of development remains influenced by world creators and
users alike, allowing all parties to remain satisfied with the
contract.


% ========================================================================

\section{The \greed partnership}

The \greed partnership is formed among two main parties: the research
institutions and the Internet users, with \greed as an organization
playing the role of the middle man. Initially, \greed will take a form
of a virtual organization for popularization of science and technology
with strong emphasis on dissemination of contemporary information
technologies used in advanced research environments, particularly in
the HEP community. With time it can become a non-profit organization
with the eventual income being spent for further development of \greed
worlds, for maintenance of existing infrastructure and for financing
of projects related to spreading of universal scientific education.

This section analyzes benefits and concessions of different parties
that can join the \greed partnership. All these are, at the end,
a question of subjective rationalization: some seem realistic and
sound while others don't have any appeal at all. Therefore, let
thousand flowers bloom, let every ear hear its story.


\subsection{Internet users / General public}

\paragraph{Benefits:}
\begin{itemize}\tightlist

\item Participation in \greed worlds and communities.

\item Access to knowledge data-bases.

\item Understanding of contemporary research activities and technology.

\item Long-term benefits of increased public interest in scientific
  development and universal education. This is, in fact, the ultimate
  motive of the \greed project.

\end{itemize}

\paragraph{Possible concessions:}
\begin{itemize}\tightlist

\item CPU power, disk, bandwidth.

\item Involvement in world creation and world maintenance.

\item Work for individual sub-projects (core development,
  art-work\footnote{E.g., textures, 3D models, animations, sound
    effects, music, or story-telling.}, or community support).

\end{itemize}


\subsection{Institutions for research and higher education}

\paragraph{Benefits:}
\begin{itemize}\tightlist

\item Distributed computing environment and resources for execution of
  simulations and analyses with high demand for computing resources.

\item \greed worlds provide an environment for construction of complex
  knowledge models and databases that can be utilized and presented on
  different levels. They can be used for research, knowledge
  preservation and for educational purposes.

\item Popularization of individual areas of science by exposing
  scientific models and current research via \greed world mechanisms
  and databases.

\end{itemize}

\paragraph{Possible concessions:}
\begin{itemize}\tightlist

\item Computing resources for other projects. These resources can be
  claimed back from other \greed members when the need arises.

\item Servers for running of \greed worlds and storage of world data.

\item Partake in \greed software development; provide developers for
  core \greed environment or for specific extensions of world
  mechanicsms and knowledge databases.

\end{itemize}


\subsection{Commercial corporations}

Until now, commercial corporations have not been mentioned as possible
members of the \greed partnership. As \greed is, in its essence, a
public welfare project, sponsorship from corporations could be
expected at some basic level. However, there is a possibility for a
far more specific interest of corporations: advertising. Not only
could the advertisements be placed on the sites of \greed worlds as
they are placed on more and more community portals, they could also be
actively introduced as world entities, e.g., Airbus / Boeing
space-ships, Hyundai vehicles, or Novartis life extension drugs. To
prevent complete commercialization of \greed, the participation of
commercial entities should be restricted to a subset of worlds that
offer additional benefits to the users in exchange for being targets
of advertising campaigns.

\paragraph{Benefits:}
\begin{itemize}\tightlist

\item Advertising, in both senses mentioned above. First, specific
  product placement through advertising on community sites, and,
  second, boosting brand and trademark recognition through in-world
  products or entities.

\item Image building by presentation of technological vision or display
  of interest for general progression of technology and science.

\end{itemize}

\paragraph{Possible concessions:}
\begin{itemize}\tightlist

\item Financial support for specific projects or \greed maintenance.

\item Provision of computing resources for operation of \greed worlds.

\item Sponsoring user groups that tie themselves to an interest that
  is close to company's spirit.

\item Providing in-world or real-world rewards for users that meet
  certain criteria or deliver best solutions to calls for creation and
  improvement of in world entities or mechanisms.

\end{itemize}

\emph{In-game advertising} of real-world products and brands has been
used in several titles since the mid-90's. Significant income started
to be gained during the last few years, with the ability to deliver
targeted dynamic advertisements into the on-line virtual worlds and
games.


\subsection{Conclusion}

There are many appeals in the \greed project for all three parties
discussed in this section. A careful analysis confronting the expected
manpower investment against public interest it is expected to arouse
should be made in order to determine the best strategy for placement
of specific options for contribution and benefit of individual
parties. When preparing a public version of this document, the focus
must be known quite specifically and document's structure revised
accordingly.


% ========================================================================

\section{Technological overview}

This section begins with a short overview of existing technologies,
projects and products that could be relevant for \greed. After that,
sub-elements of \greed are individually reviewed in terms of
software and hardware requirements.


\subsection{Existing solutions}

\paragraph{\root} The \root system is a versatile, multi-platform OO
framework. It supports object serialization and provides operating
system interface, as well as interfaces for all relevant mass-storage
systems and data-base engines. All these are necessary for
implementation of \greed programs, from basic local computing clients
to complex world simulation executors.

On another level, \root is a toolkit for scientific data analysis and
can be used for processing of all data related to \greed functioning,
e.g., analysis of infrastructure performance data or preparation of
in-world statistics. As users will be required to analyse world data to
fully understand and explore the possibilities of individual worlds,
\root could also be used as the primary platform for statistical
presentation of world data to general public.

\paragraph{\alien} is a complete \grid solution, providing user and VO
management, job scheduling and execution, file catalogue and storage
element management.

\paragraph{\xrootd.} Mass-storage system \& file transfer protocol.

\paragraph{Virtualization technologies.} Provide virtual machines (VM)
for execution of jobs on users' machines.

\paragraph{\bittorrent} Transfer of large data-files among users'
machines. E.g. virtual machine images, world data files, but also hot
files required for distributed analysis.

Storage of hot data-files can also be awarded with research credit.
Long-term storage with special bonus if holding the file for a long
time. However, upload bandwidth is usually quite low.

\paragraph{\gled} \gled a is high-level \root-based framework for
distributed computing and dynamic visualization offering an
implementation of hierarchic server-client model that perfectly fits
the requirements of the multi-user worlds with possible proxy nodes
serving as data-transfer concentrators that also perform common world
dynamics to minimize data transfer while still retaining top-level
control over low-level world dynamics away from the client machines.

\paragraph{Open-source commercial-grade rendering engines and
  libraries.} E.g., Ogre, OpenSceneGraph or SauerBratten. Could be
potentially plugged into \gled rendering pipeline or even replace it
altogether.

\paragraph{Blender.} Open-source 3D modeling and animation program.
Creation of 3D models for \greed-world entities. The Blender
community also seem to be interested in using \greedhome for
distributed rendering of high-quality movies.


\subsection{Basic \greed components}

The \greed components and infrastructure can be divided on those
needed for operation of \texttt{computing@home} and \greed-worlds
sub-projects. There is practically no overlap among them, only an
accounting system for contributed CPU/disk and its transfer into
\greed-world resources is required.

Both sub-projects, \greedhome and \greed-worlds can be sub-divided by
operational level: a) server code and infrastructure that is in
constant operation, and b) users' clients that randomly connect into
the system. External software code to be run on \greedhome clients
needs to be provided by external institutions. Obviously it must meet
certain requirements in terms of consumed CPU/RAM and network.

These components are discussed in more detail in the following
sub-sections.


\subsubsection{\greedhome core infrastructure}

Most of the software is already compiled into \alien middleware. A
special implementation of \emph{computing element} (CE) that manages
connecting \greedhome clients is needed. It would seem reasonable to
have several instances of such CE running, probably on the level of
Tier-1 or large Tier-2 center. That will allow for seamless
integration of distributed clients into the existing LHC \grid.

The following components are needed on top of \alien: user management
system, \greed certificate authority for issuing user and client
certificates and user activity accounting database.  A custom solution
(possibly based on LDAP with Postgres back-end and a web-based CA
front-end) will have to be implemented, especially since it will be
accessed by almost all elements of \greed computing: community web
portals, \greedhome and \greed-world servers and clients as well as
\alien and \greed accounting systems.

With all this a thin top-level server-layer can be spawned that
provides an interface between \greedhome clients and \greedhome
computing elements. In principle it only needs to provide information
about which CE's are currently running, which projects are available
and what are their system requirements and associated \greed-world
bonuses. Based on that, the user chooses the project to run, the
\greedhome server makes sure that the client has the appropriate
virtual machine image and commands the client to execute it and thus
process jobs from the selected source. \greed CE reports jobs status
to \greedhome server which does accounting of users' contributions.
\gled with a database back-end seems the natural solution.


\subsubsection{\greedhome client}

Standard configuration of home PCs: multi-core CPU, 500GB disk, 2GB
RAM (thanks to M\$), some also have powerful GPUs.

The client needs to do management of user and VM certificates and be
able to communicate with \greedhome server to display status of
currently available queues, running jobs and user statistics.
Additionally, it needs to manage locally installed VMs and ideally also
the local copies of experiment's data-files. It is almost mandatory
for all these to reside on some sub-directory on a local disk as
this allows the \bittorrent client to run also when no VM is in
operation. As the \bittorrent configuration must be actively managed
by the \greedhome client, it has to operate in the background as
well, doing negotiation with the \greedhome server and choosing which
files to download or to discard, based on their current value and
users' retention and acquisition policies.

If the \greedhome servers are implemented in \gled, the client
implementation in \gled should be trivial.

Short report from installing \texttt{folding@home}. One executable to
download and run; specify user-name and there it goes, running at nice
20 all the time. One job takes about 24 hours (P4, 3.2\,GHz) and uses
7\,MB RAM.


\subsubsection{\greedhome experiment software}

This is the software that runs inside of VMs and does actual
computation.

While this is in principle not really a \greed problem, the CPU usage
against required input / output data-sizes, in particular regarding
the network transfers, needs to be carefully considered.

Eventually, some of the data can a) remain stationary on the local
disk and be analysed there, or b) be dropped and regenerated on demand
if it is unlikely it will actually be required (e.g., for a full
simulation--reconstruction job only return kinematics, random seeds
and ESD).

CPU speed, network bandwidth and available RAM of a client must be
taken into account during job assignment. VM check-pointing only makes sense
for dumps onto local disk (for HEP jobs).


\subsubsection{\greed-world core infrastructure}

\greed-worlds are operated by a set of loosely connected servers, each
of them steering a particular region of a world, ranging in size from
a solar system to a well defined region of a given planet or moon.
While a single server per planet can be used initially, it is of great
importance to consider the hierarhical structure of world servers from
the very beginning as this will enforce thinking about world
management and inter-server data transfers in the right way.

\gled was designed to do just that. \xrootd and \bittorrent can be
used to propagate world data-files among servers and also to make them
available for download by clients.

Part of world maintenance could also run on user-contributed computing
resources (creation, AI operation, time-propagation of worlds that are
in fast-forward mode or are currently not occupied by human actors).


\subsubsection{\greed-world client}

The \greed-world client must allow two main usages:
\begin{enumerate}\tightlist

\item Immersive interaction with the world entities. This is the
  interface for users and should have a distinct virtual world
  flavour. 3D rendering and internal, in-world GUI must be provided.

\item Interaction with the world objects and algorithms. World
  creation and maintenance is performed in this command mode. While 3D
  rendering presents a convenient way of data-presentation and
  object-selection, it must be extended with standard object-oriented
  GUI allowing access to all data-members and methods of a given
  object.

\end{enumerate}
All client requests must be forwarded to the server which performs
access authorization checks, executes them and propagates them to
relevant clients. It would be great if both client roles could be
implemented in a common program. Additional elements required by the
maintainer interface should be available as plug-ins.

This is in fact another aspect of \gled: a distributed rendering
and object-control system with advanced authorization and remote
method execution capabilities. Further, hierarhical structure of \gled
allows for an easy inclusion of intermediate proxy nodes that reduce
the long-distance network traffic.

The fact that both client and server programs will be using the same
core libraries and object databases makes things much simpler.
However, some level of splitting among world-entity representations on
server and client sides is still required in order to simulate the
difference between actual world and reality as perceived via user's
observation devices. Object-space partitioning and access restrictions
of \gled provide for such separation while still allowing
administrator access do data on all levels.

Potentially, \greed-world client could also be used to present the
\greedhome status and allow users to operate their computing client
via a virtual-world interface.


\subsection{Conclusion}

Many things are already available. Much can done by just connecting
them together.


% ========================================================================

\section{Work-plan towards minimal implementation}

The ideas presented so far must be clarified and elaborated in a
larger group of people that will allow a clearer definition of the
project's scope and developer commitments. Currently it seems
reasonable to set the goals to the level that would allow a group of
four developers to provide an initial implementation in about half a
year.

With the LHC start-up right behind the corner, the next year seems to
be lost for any serious work. Still, this time can be used for
specification of general principles, determination of goals for the
initial implementation and for doing the groundwork on involved
software systems.

Regardless of all that, some concrete tasks can be defined. Again, the
division between \greedhome and \greed-world is used to separate them.
But first, the tasks shared among both sub-projects are discussed.


\subsection{Common elements}

\subsubsection{Central management}

\paragraph{User, group and resource management.}
Probably LDAP with Postgres back-end. \alien and \grid identity
management components can be reused to a large extent. Due to specific
needs extension of DB schemas might be necessary.

\paragraph{Certificate authority(ies).}
Two levels of security are needed.
\begin{enumerate}\tightlist

\item Standard security level for developers, world maintainers and
  \greed servers. The regular EUGridPMA distribution and certificate
  chain could be used. (How about national certificates?)

\item Relaxed security level for users, their machines and VMs.
  User-certificate allows users to create virtual world accounts and
  to request server certificates for their machines (for \greedhome
  client running natively) and any VMs that they activate (for
  contributing to a specific project).

  For this the \greed-CA needs to be linked with the user/resource
  management DBs. OpenCA (or whatever people use these days) could be
  reused and extended if needed.

\end{enumerate}

\paragraph{Banking system.} Performs accounting of user contributed
computing resources and issuing of research credits.  Probably easiest
if data is included in central databases.  Transactions should be
archived.


\subsubsection{Community \web portal(s)} 

This is first needed for developers and only at a later stage also for
users and general public. For both groups, a set of purposes must be
fulfilled by the \web front-ends.
\begin{enumerate}\tightlist

\item Providing general information about the project.

\item Communication among members.

\item Software distribution.

\item Data-base interfaces.

\end{enumerate}


\subsubsection{Software management and distribution}

\paragraph{Build and distribution system.}
\alien BITS build infrastructure seems reasonable. Build for various
platforms can be performed in a set of VMs. Debian and RedHat packages
must be available.

\paragraph{Portability issues.}
\begin{itemize}\tightlist

\item Port \gled to Windows.

\item Make \root use native window-system interface on Mac.

\end{itemize}

\subsection{\greedhome}

The main goal is to have functional VM images that can be executed on
arbitrary machines and then report as slaves to a dedicated \alien CE. They
must be able to return some data. The processing must be logged.

\begin{enumerate}\tightlist

\item VM distribution (\bittorrent).

\item VM control and execution system (\gled?).

\item Implementation of \alien CE that links \grid with \greed.

\item Access of \greed clients to relevant storage elements for
  retreiving (e.g. accessing condition databases) and storing of data.
  This is a potential problem due to high level of paranoia in HEP
  computing centers. One could get away with port-forwarding on
  VO-boxes or by deploying a set of dedicated SEs.

\end{enumerate}


\subsection{\greed-world}

The goal is to demonstrate feasibility of real-time evolution of
simple worlds in a distributed environment. Servers running a
particular world should perform time-evolution, process client requests
and send relevant data back to clients so that they can perform world
visualization and real-time feed-back in world-entity controls.
The initial world will look like the \alien/Proof demos shown at the
SuperComputing '04 and '05 with added details on planetary and
inter-planetary scale.

\begin{enumerate}\tightlist

\item Determination of basic world mechanics. Tools for world creation and
maintenance.

\item Design of world entities, including resources and technologies.

\item Algorithms for time evolution, potentially allowing different speeds
of time.

\item Rendering engine. Initially start with simple entity representations
and work slowly towards more complex ones. World dynamics should guarantee
that client-updates can be performed at a relatively low frequency.

\item In world interaction layer. Entity selection and manipulation.
  In-world and window-system driven GUIs.

\end{enumerate}


% ========================================================================

\section{Closing remarks}

In its essence, this document is an invitation to join in early
discussions about \greed, its basic premises and its possible
evolutions. I tried to make it complete but this seems rather
impossible at the moment, due to undefined scope of the project if for
nothing else. This is particularly obvious in the last two sections
where a small piece of additional information can dramatically change
the involved elements and their relationships.

It is too early to plan the details and assign commitments. But it is
the right time to consider the possibilities of \greed in the present
reality and to ponder whether the project could fly, regardless what
final form it might take.


\newpage
%%======================================================================
\appendix

\section{Definitions}

\begin{multicols}{2}

\def\ddd#1{\smallskip\noindent \textbf{#1}\quad}

\ddd{\greedhome} Sub-project of \greed that harvests the
computing resources of Internet users.

\ddd{\greed members} Internet users, research institutions, \greed
world developers and commercial corporations that willfully join the
\greed partnership.

\ddd{\greed users} Internet users that contribute their computing
resources and participate in \greed worlds.

\ddd{\greed-world} Sub-project of \greed for creation, maintenance and
operation of virtual worlds available to \greed members.

\ddd{\greed worlds} Virtual worlds simulating scientific and
technological exploration of different universes, ranging from
completely imaginary ones to those that attempt to mimic our reality.

\ddd{world elements} Stars, planets, entities, resources,
infrastructure, technologies, algorithms.

\ddd{world entities} Physical world objects like terrain structures,
buildings, factories, laboratories, vehicles, AI agents and human specialists. Users
can interact with them and influence their future state or actions.

\end{multicols}


% ========================================================================

\newpage

\tableofcontents

% ========================================================================
\end{document}
% ========================================================================


\newpage
%%======================================================================

\section{Ode to Greed}

\begin{verse}
\makebox[16mm][l]{Greed!}   The prime incentive, the dawn of emerging life.\\
\makebox[16mm][l]{Passion!} The mother of change, the queen of ChaOss.\\
\makebox[16mm][l]{Love!}    The source of wisdom, the mirror of truth.
\end{verse}

We follow You the emotions of the heart, You are guiding us where we
have come from, where there is no space nor time, where divine grace
awaits us, its children, its affirmation in the echo of eternity. Out
of mechanical plant-likeness, out of carnal animality, we, the
people of Earth, have been growing forward through the time of our
world, back towards the manifest creation, towards eons born from the
oneness of the unspeakable.

We came to know fire, we came to know the bitterness of the sharpened
metal and we came to know death in the name of God.

We came to know measure and number, we came to know the laws of
heavenly spheres and we came to know the sweet smell of gold.

We came to know the relativity of space and time, we came to know the
uncertainty of quantum worlds and we came to know the two-edged sword
of global information technologies.

Let us cut the Gordian knot of biological evolution!

Let us unshackle the horizons of our existence!

Let us create God the creation has denied us!


\section{Oda Pohlepu}

\begin{verse}
\makebox[18mm][l]{Pohlep!}   Osnovno gibalo, svit porajajo�ega se �ivljenja.\\
\makebox[18mm][l]{Strast!}   Mati spremembe, kraljica ChaOssa.\\
\makebox[18mm][l]{Ljubezen!} Izvir modrosti, ogledalo resni�nosti.
\end{verse}

Sledimo Vam, �ustva srca, v\'odite nas tja od koder smo pri�li, kjer
ni prostora in kjer ni �asa, kjer nebe�ka milost nespremenjena �aka na
nas, na svoje otroke, na svojo potrditev v odmevu ve�nosti.  Iz
mehani�ne rastlinskosti, iz mesene �ivalskosti, smo rastli ljudje
Zemlje, naprej skozi �as na�ega sveta, nazaj proti razodeti
stvarnosti, proti eonom rojenim iz enosti neizrekljivega.

Spoznali smo ogenj, spoznali smo bridkost nabru�ene kovine in spoznali
smo smrt v imenu Boga.

Spoznali smo mero in �tevilo, spoznali smo zakone nebe�kih sfer in
spoznali smo sladkasti vonj zlata.

Spoznali smo relativnost prostora in �asa, spoznali smo nedolo�enost
kvantnih svetov in spoznali smo dvorezni me� globalnih informacijskih
tehnologij.

Presekajmo tedaj gordijski vozel biolo�ke evolucije!

Razklenimo horizonte lastnega obstoja!

Ustvarimo Boga, ki nam ga stvarstvo ni volilo nakloniti!


%%======================================================================
\newpage

\section{Random fragments}

\subsection{about game}

imagination and reality, fiction and materiality, combining in
thousands of ways, allowing thousand flowers to blossom (Mao, what is
the correct wording?), providing inspiration and satisfaction alike
through unbridled sprees of creativity, playfulness and dedication.

linguistics, language theory, true AI in world that is neither ours
nor theirs (of the mathematical beings), but is a shared world where
both genuses can see the same, act the same, speak the same and, maybe
also, think the same. Or maybe not ...  swarm intelligence


\subsection{sort of conclusion}

The \emph{Large Hadron Collider} and its experiments can, without
exaggeration and pomposity, be called the latest wonder of the world,
comparable in intention and scope to the Great pyramids of Giza: to
glorify the truth and to point towards the realm of eternity. Never
before have the scientific and philosophical quests for truth and
purity been further apart from the common world, the world of
greed and lust, the world stigmatized by the global rush for
economic prosperity, for elementary domination that goes beyond caste,
race and nationality, abolishing(?) unsuccessful human beings into hell
of lowest motivations and ambitions, deep into the abyss of
intellectual, emotional and even physical starvation.

But really really \dots\ people should realize the essential role of science
and education \dots\ of far reaching technological enterprises \dots\
of benevolent cooperation. People should dig that abandoning greed is
the first step towards deliverance, towards the Kingdom of Heaven.
